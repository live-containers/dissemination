\chapter{Background Concepts} \label{chap:background}

\section{Containers} \label{sec:containers}

Jordi, you might add comments using \texttt{\textbackslash jg} command \jg{like this}.

An Introduction to containers, what are they used for, some data, ...

\subsection{An Introduction to Virtualization}

Introduce the concepts of virtualization and process isolation, what is a virtual machine

\subsection{Working Principles of Containers}

Relatively deep dive into containers:

\subsubsection*{Control Groups}

\subsubsection*{Namespaces}

\subsubsection*{Putting it all together}

Mention the different alternatives for container runtimes

\subsection{Container Orchestrators}

\subsection{State-of-the-Art for Virtualization Techniques}

Mention some novel techniques in the area:

- Microkernels

- Lightweight VM

- Serverless

- Cloud native

\section{Checkpointing}

Brief introduction, what is checkpointing

\subsection{Checkpoint-Restore}

\subsection{Live Migration}

Iterative, pre-copy, post-copy, ... \cite{Clark2005}

\subsection{Distributed Checkpointing}

\section{\criu: Checkpoint Restore in Userspace}

Given how heavily our project relies on CRIU, I figured out we could devote a section to it.
Here I will depict the architecture, how it works, and who uses it, together with some of it's most important features.

\subsection{A Technical Overview on \criu}

\subsection{Comparison with Other C/R Tools}

We can beneffit from this table:

- \url{https://criu.org/Comparison_to_other_CR_projects}

Compare:

- VM C/R

- CRIU

- DMTCP

- BLCR

- FTI: maybe this falls a bit out of the scope \url{https://github.com/leobago/fti}

We could also include snippets of how to run a Hello World for each alternative.

