%% ENGLISH VERSION ------------------------------------------------------------
\addcontentsline{toc}{chapter}{Abstract}
\topskip0pt
\vspace*{1.50cm}
\begin{center}

    \LARGE
    \textit{\textbf{Abstract}} \label{sec:abstract}

    \vspace{0.5cm}

    \large
    \textbf{\titleEN}

    by \textsc{Carlos Segarra Gonz\'alez}
\end{center}

\vspace{0.5cm}

\normalsize
Containers have become the go-to technology for managing application's lifecycle in the cloud.
As a consequence, cloud-tenants are becoming increasingly interested in advanced load-balancing strategies to optimize resource usage and guarantee good quality of service.
Live migration is a technique to halt the execution of a program and resume it in the same state in a different location.
It relies on checkpoint-restore tools to snapshot an application's state.
Checkpoint-Restore in Userspace (CRIU) is one of such tools, designed to work transparently to the user, entirely from userspace, and specialized for containers.

In this Master thesis we present a tool to perform live migration of \runc containers using \criu.
Our solution is efficient in terms of resource utilization, memory and disk, and minimizes downtime when compared to naive migration through checkpoint-transfer-restore and native virtual machine (VM) migration.
We also provide support to checkpoint memory and network intensive containers with established TCP connections and external namespaces.
The implementation is accompanied by a thorough background research, together with a set of micro benchmarks to justify each of our design choices.
It is open sourced and available in the project's repository.

Our evaluation results show that, by adding a very small overhead ($0.1$s to the baseline of checkpoint-transfer-restore) we improve scalability with regard to allocated memory by a factor of 10.
Additionally, all our results are an order of magnitude faster than traditional virtual machine migration.
Lastly, our benchmarking of network intensive server-side application's migration reported a $< 0.1$s throughput downtime, negligible with more moderate workloads.
As a consequence, we believe our migration technique for \criu and \runc is a feasible replacement for VM migration and, as the technology matures, will be ready for deployment in production.

\vspace{0.5cm}

\textbf{Keywords:} checkpoint, restore, live migration, CRIU, runc, container, load-balancing

\vfill
\pagebreak

%% CATALAN VERSION ------------------------------------------------------------
\topskip0pt
\vspace*{1.5cm}
\begin{center}
    \LARGE
    \textit{\textbf{Resum}} 

    \vspace{0.5cm}

    \large
    \textbf{\titleEN}

    per \textsc{Carlos Segarra Gonz\'alez}
\end{center}

\vspace{0.5cm}

\normalsize

Els contenidors de programari han esdevingut la tecnologia referent per gestionar aplicacions al n\'uvol.
Com a conseq\"u\`encia, els principals prove\"idors de servei estan cada vegada m\'es interessats en solucions per gestionar dits contenidors, i oferir garanties de qualitat als seus usuaris.
La migraci\'o d'aplicacions consisteix en aturar un proc\'es i reiniciar-lo a un altre entorn d'execuci\'o en el mateix punt en el que s'havia aturat.
Es basa en la capacitat de generar captures de l'estat d'execuci\'o d'un proc\'es.
\textit{Checkpoint-Restore in Userspace (CRIU)} \'es un projecte de codi obert que permet obtenir dites captures de manera transparent a l'usuari, sense modificar-ne el kernel, i especialitzada en contenidors.

En aquesta t\`esis de M\`aster, presentem una eina per realitzar migracions de contenidors tipus \runc emprant \criu.
La nostre soluci\'o \'es eficient en termes d'utilitzaci\'o de recursos, mem\`oria i disc, i minimitza el temps de migraci\'o quan comparada amb una migraci\'o basada en capturar-transferir-reiniciar i amb la migraci\'o nativa de m\`aquines virtuals oferida pels seus prove\"idors.
En afegit, la nostra eina permet migrar aplicacions que fan \'us intensiu tant de mem\`oria com de xarxa, amb connexions TCP establertes, i \textit{namespaces} externs.
La implementaci\'o est\`a acompanyada d'una recerca bibliogr\`afica en profunditat, aix\'i com d'una s\`erie d'experiments que motiven els nostres criteris de disseny.
El codi \'es de lliure acc\'es i es pot trobar a la p\`agina web del projecte.

Els nostres resultats mostren que, afegint una petita redund\`ancia ($0.1$s al temps de refer\`encia de capturar-transferir-reiniciar) millorem l'escalabilitat del sistema en termes d'utilitzaci\'o de mem\`oria en un factor 10.
En afegit, tots els nostres resultats s\'on un ordre de magnitud m\'es r\`apids que les migracions tradicionals de m\`aquines virtuals.
Per \'ultim, els nostres experiments amb aplicacions que fan \'us intensiu de xarxa mostren una caiguda del servei inferior als $0.1$ segons, imperceptible per clients amb c\`arregues de treball m\'es moderades.
A mode de conclusi\'o, creiem que la t\`ecnica de migraci\'o que proposem en aquest projecte per \criu i \runc \'es una alternativa viable a la migraci\'o de m\`aquines virtuals i, a mesura que la tecnologia maduri, estar\`a llesta per entorns de producci\'o.

\vspace{0.5cm}

\textbf{Paraules Clau:} migracio, contenidor, CRIU, runc, captura d'estats, \textit{checkpoint}

\vfill
