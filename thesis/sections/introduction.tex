\chapter{Introduction} \label{chap:introduction}

\cs{This section still needs to be phrased together}
\jg{This needs a good elaboration of the text. My feeling now is that there is some mixing among objectives and tasks here. We should detail both, but providing a clear rationale. For instance, accomplishing the goal 'Minimize downtime and resource utilization of memory-intensive applications' requires a task on 'supporting iterative migration'. Similarly, the task 'Support for established TCP connections' is necessary because one of the goals is the 'ability to live migrate distributed container deployments'.}

Introduction/motivation

The main goal of this work is to implement efficient live migration of running containers.
The terms \textit{efficient} and \textit{live} are vague in the absence of concise metrics, and the variety of running containers is also huge, as a consequence we specify a set of objectives we want our system to fulfill.
\begin{itemize}
    \item Minimize downtime and resource utilization of memory-intensive applications.
    \item Support for established TCP connections
    \item Support for C/R of namespaces
\end{itemize}

The contributions of this work are the following:
\begin{itemize}
    \item An exhaustive micro-evaluation of different \criu features, their performance, and their integration with \runc.
    \item An open-source library for live migration of \runc containers using \criu.
    \item An evaluation of our solution when compared to virtual machine migration.
\end{itemize}

The structure of the rest of this document is as follows:
