\vspace*{2cm}
\epigraph{Like most of my generation, I was brought up on the saying: 'Satan finds some mischief for idle hands to do.' Being a highly virtuous child, I believed all that I was told, and acquired a conscience which has kept me working hard down to the present moment. But although my conscience has controlled my actions, my opinions have undergone a revolution. I think that there is far too much work done in the world, that immense harm is caused by the belief that work is virtuous, and that what needs to be preached in modern industrial countries is quite different from what always has been preached. Everyone knows the story of the traveler in Naples who saw twelve beggars lying in the sun (it was before the days of Mussolini), and offered a lira to the laziest of them. Eleven of them jumped up to claim it, so he gave it to the twelfth. This traveler was on the right lines. But in countries which do not enjoy Mediterranean sunshine idleness is more difficult, and a great public propaganda will be required to inaugurate it. I hope that, after reading the following pages, the leaders of the YMCA will start a campaign to induce good young men to do nothing. If so, I shall not have lived in vain.}{Bertrand Russell, \textit{In Praise of Idleness}}
\vfill
\pagebreak
