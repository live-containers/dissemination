%%%%%%%%%%%%%%%%%%%%%%%%%%%%%%%%%%%%%%%%%%%%%%%%%%%%%%%%%%%%%%
%% Carlos Segarra's Beamer Presentation Template. All credits
%% to Vincent Labatut from whom I took the template and added
%% my own flavour to it. Kudos to <vincent.labatut@univ-avignon.fr>
%%%%%%%%%%%%%%%%%%%%%%%%%%%%%%%%%%%%%%%%%%%%%%%%%%%%%%%%%%%%%%
% setup Beamer
\documentclass[9pt,    % default is 11pt, use 10pt for more compact slides
%    handout,            % collapse all overlays (=animations) and video-invert console text
    english,            % presentation language (theme supports only french & english)
    xcolor=table,       % colors in the tables
    envcountsect,        % include section number in theorem numbers
    aspectratio=169     % Using 16:9 aspect ratio because 2019
]{beamer}

%%%%%%%%%%%%%%%%%%%%%%%%%%%%%%%%%%%%%%%%%%%%%%%%%%%%%%%%%%%%%%
% setup the theme
%\usepackage{au/sty/beamerthemeAU}         % no option at all
\usepackage[light]{csg-temp/sty/beamerthemeAU}   % the "light" option only changes the title and section pages

%%%%%%%%%%%%%%%%%%%%%%%%%%%%%%%%%%%%%%%%%%%%%%%%%%%%%%%%%%%%%%
% setup side notes
\usepackage{pgfpages}                                   % comment all 3 below lines to hide notes
%\setbeameroption{show notes}                           % alternate content and note slides
%\setbeameroption{show only notes}                      % only note slides
%\setbeameroption{show notes on second screen=right}    % dualscreen: right, left, top, bottom
%\usepackage{enumitem}

%%%%%%%%%%%%%%%%%%%%%%%%%%%%%%%%%%%%%%%%%%%%%%%%%%%%%%%%%%%%%%
% name of the biblatex file
\addbibresource{biblio.bib}

%%%%%%%%%%%%%%%%%%%%%%%%%%%%%%%%%%%%%%%%%%%%%%%%%%%%%%%%%%%%%%
% External Packages
\usepackage{datenumber}
\usepackage{varwidth}
\usepackage[normalem]{ulem}

% Math Mode
%\usepackage{algorithm}
%\usepackage{algorithmic}
%\usepackage{algorithm2e}
%\usepackage{multicol}
%\usepackage[noend]{algpseudocode}

%%%%%%%%%%%%%%%%%%%%%%%%%%%%%%%%%%%%%%%%%%%%%%%%%%%%%%%%%%%%%%
% title and subtitle of the presentation (the latter is optional)
\newcommand{\mainTitle}{Checkpoint/Restore of Established TCP Connections}
\newcommand{\secondTitle}{Design, Implementation, and Evaluation on runC Containers}
\subtitle{Decentralized Systems} % leave empty if no subtitle
\title[C/R of Established TCP Connections] % leave empty for no title in footer
    {\normalsize Master in Advanced Mathematics and Mathematical Engineering \\[5pt] \Large Transparent Live Migration of Container \\ \Large Deployments in Userspace}
\subtitle{Master's Thesis - Oral Deffense} % leave empty if no subtitle
%\subtitle{Master in Research in Informatics - MIRI}
%%%%%%%%%%%%%%%%%%%%%%%%%%%%%%%%%%%%%%%%%%%%%%%%%%%%%%%%%%%%%%
% date of the presentation (leave empty for no date, default is today)
\date[July 7, 2020] % leave empty for no date in footer
    {Barcelona, Tuesday July 7, 2020}
    %{\datedayname, \today}
%%%%%%%%%%%%%%%%%%%%%%%%%%%%%%%%%%%%%%%%%%%%%%%%%%%%%%%%%%%%%%
% authors and their affiliations (the latter is optional)
\author[] % leave empty for no author in footer
{Carlos Segarra - \texttt{carlos.segarra@estudiant.upc.edu} \\ \textit{Advisor:} Jordi Guitart - \texttt{jguitart@ac.upc.edu}}
%{\inst{1} Computer Science Lab, Avignon University -- LIA EA 4128 \texttt{\{firstname.lastname\}@univ-avignon.fr}
%\and \inst{2} Institute of Disruptive Innovation, University of Excellence \texttt{\{firstname.lastname\}@univ-excell.fr}
%}
%%%%%%%%%%%%%%%%%%%%%%%%%%%%%%%%%%%%%%%%%%%%%%%%%%%%%%%%%%%%%%
% optional: additional logo (ex. lab)
%\titlegraphic{\includegraphics[width=3cm,]{images/logo_FME.png}}
% if you want several logos, put them in a box
%\titlegraphic{\parbox{3cm}{\includegraphics[width=3cm,]{images/ceri_logo.pdf}\newline\includegraphics[width=3cm,]{images/lia_logo.pdf}}}
%%%%%%%%%%%%%%%%%%%%%%%%%%%%%%%%%%%%%%%%%%%%%%%%%%%%%%%%%%%%%%

%%%%%%%%%%%%%%%%%%%%%%%%%%%%%%%%%%%%%%%%%%%%%%%%%%%%%%%%%%%%%
% Presentation speciphic packages
% \usepackage{multicol}
% \usepackage[titles]{tocloft}
% \renewcommand{\cftchapfont}{\normalfont\bfseries}
\usetikzlibrary{decorations.pathmorphing, patterns}
\usepackage{tabularx}
\newcolumntype{L}[1]{>{\raggedright\arraybackslash}p{#1}}
\newcolumntype{C}[1]{>{\centering\arraybackslash}p{#1}}
\newcolumntype{R}[1]{>{\raggedleft\arraybackslash}p{#1}}
%%%%%%%%%%%%%%%%%%%%%%%%%%%%%%%%%%%%%%%%%%%%%%%%%%%%%%%%%%%%%

\usepackage{multimedia}
%%%%%%%%%%%%%%%%%%%%%%%%%%%%%%%%%%%%%%%%%%%%%%%%%%%%%%%%%%%%%%
\begin{document}
%%% title page
%% Outline of the presentation
% 1. Introduction - 1 slide  (include contribution list!)
% 2. Background Concepts
%   2.1. Containers - 1 slide
%   2.2. Checkpointing - 1 slide
%   2.3. CRIU - 1 slide
% 3. Implementation
%   3.1. Design Choices
%     3.1.1. Diskless Migration - 1 slide
%     3.1.2. Iterative Migration - 1 slide
%     3.1.3. TCP Connections - 1 slide
%   3.2. Implementation Details - 2 slides
% 4. Evaluation
%   4.1. Application Downtime - 1 slide
%   4.2. Scalability regarding memory size - 1 slide
% 5. Conclusions & Future Work - 1 slide

\begin{frame}
  \titlepage
\end{frame}

\begin{frame}
    \frametitle{Introduction}
    \framesubtitle{Framing the Work}

    \textbf{\textcolor{blue}{Problem Motivation:}}
    \begin{itemize}
        \item \textbf{\textcolor{blue}{Containers}} have become the go-to alternative for sandboxing and deploying distributed applications, they build on the concept of \textbf{\textcolor{blue}{namespaces}}.
        \item \href{https://en.wikipedia.org/wiki/Application_checkpointing}{Checkpointing} provides systems with \textbf{\textcolor{blue}{fault-tolerance}} and \textbf{\textcolor{blue}{state consistency}} enabling rollback and restore from previous stable versions.
        \item \textbf{\textcolor{blue}{Checkpoint-Restore in Userspace (CRIU)}} is a tool to perform checkpoint-restore of processes, transparently to the user.
        \item \textbf{\textcolor{blue}{Live migration}} consists in transparently relocating running services for improved resource-management and increased QoS.
    \end{itemize}

    \begin{columns}
        \begin{column}{.1\textwidth}
        \end{column}\hspace{-5cm}
        \begin{column}{.7\textwidth}
            \begin{alertblock}{Main Goal}
                Design, implement, and evaluate a tool for efficient live migration of running containers using \textbf{CRIU} and \textbf{runC}.
            \end{alertblock}\hfill
        \end{column}
    \end{columns}

\end{frame}

\begin{frame}
    \frametitle{Background Concepts}
    \framesubtitle{Containers and runC}

    \only<1>{
        \vspace{-15pt}

        \textcolor{blue}{\textbf{Linux Containers:}}
        \begin{alertblock}{Definition}
            A linux container is a set of processes isolated from the rest of the system.
        \end{alertblock}
        \begin{itemize}
            \item They rely on \textbf{\textcolor{blue}{namespaces}} and \textbf{\textcolor{blue}{control groups}} for fine-grained resource control.
            \item A \textbf{\textcolor{blue}{container engine}} transforms a \emph{container image} into a running process relying on a chosen \emph{runtime}.
            \item Some well-known container engines are \textsc{Docker}, \textsc{Podman}, and \textsc{Cri-O}.
        \end{itemize}
    }

    \only<1-2>{
        \textbf{\textcolor{blue}{Introduction to runC:}}
        \begin{itemize}
            \item \textbf{\textcolor{blue}{runc}} is the Open Container Initiative's (OCI) reference runtime implementation.
            \item Rather than \textbf{\textcolor{blue}{images}}, it relies on OCI bundles: \texttt{config.json} + \texttt{rootfs}.
            \item It sits at the core of most container engines as depicted in Figure 1.
        \end{itemize}
    }

    \only<2>{
        \begin{figure}[h!]
            \includegraphics[width=.5\textwidth]{./images/runc.png}
            \caption{Placing runC in the call stack of different container engines.\label{fig:runc}}
        \end{figure}
    }

\end{frame}

\begin{frame}
    \frametitle{Background Concepts}
    \framesubtitle{Checkpoint-Restore and Live Migration}

    \textcolor{blue}{\textbf{Checkpoint-Restore:}}
    \begin{alertblock}{Definition (Encyclopedia of Parallel Computing)}
        Checkpointing refers to the ability to store the state of a computation in a way that allows it be continued at a later time without changing the computation's behavior.
    \end{alertblock}

    \begin{columns}[t]
        \begin{column}{.65\textwidth}
            \begin{itemize}
                \item The \textbf{\textcolor{blue}{state}} is made up of: memory, pipes, sockets, ...
                \item This state is used to \textbf{\textcolor{blue}{restore}} the process.
                \item Originated in HPC and popularized through Virtual Machines.
                \item \textbf{\textcolor{blue}{Live migration}} consists in checkpointing in an environment, and restoring in a different one.
            \end{itemize}
        \end{column}\hfill
        \begin{column}{.35\textwidth}
            \vspace{-18pt}
            \begin{figure}
                \centering
                \includegraphics[width=.8\textwidth]{./images/cr.png}
                \caption{Basic working principle of live migration.}
            \end{figure}
        \end{column}
    \end{columns}

\end{frame}

\begin{frame}
    \frametitle{Background Concepts}
    \framesubtitle{\textsc{CRIU}: Checkpoint-Restore in Userspace}

    \begin{columns}[t]
        \begin{column}{.65\textwidth}
            \textbf{\textcolor{blue}{Checkpoint/Restore of Unix sockets:}} 
            \begin{itemize}
                \item We rely on \textbf{\textcolor{blue}{CRIU}} to perform C/R from userspace, \textbf{\textcolor{blue}{transparently}} to the user.
                \item To \textbf{\textcolor{blue}{checkpoint}}, CRIU gathers information from FDT and dumps the content.
                \item To \textbf{\textcolor{blue}{restore}}, it depends on the socket state:
                \begin{itemize}
                    \item \texttt{LISTEN}: restore as any other file.
                    \item \texttt{CONNECTED}:
                    \begin{enumerate}
                        \item Establish a listening anon. socket.
                        \item Create a socket at the desired FD (dumped one).
                        \item Connect to the fake end and call \texttt{dup2}.
                    \end{enumerate}
                \end{itemize}
            \item We need to bind to the \textbf{\textcolor{blue}{same}} IP address.
            \item Inbound packets are dropped by the netfilter in Table~\ref{table:filter}.
            \end{itemize}
        \end{column}
        \begin{column}{.40\textwidth}
            \vspace{-18pt}
            \begin{table}
                \centering
                {\ttfamily \scriptsize
                \begin{tabular}{p{0.7cm}p{0.3cm}p{0.3cm}p{0.5cm}p{0.5cm}p{1.2cm}}
                    \multicolumn{6}{l}{Chain INPUT (policy ACCEPT)} \\
                    target & prot & opt & source & dest & \\
                    CRIU & all & -- & .../0 & .../0 & \\
                    & & & & & \\
                    \multicolumn{6}{l}{Chain FORWARD (policy ACCEPT)} \\
                    target & prot & opt & source & dest & \\
                    & & & & & \\
                    \multicolumn{6}{l}{Chain OUTPUT (policy ACCEPT)} \\
                    target & prot & opt & source & dest & \\
                    CRIU & all & -- & .../0 & .../0 & \\
                    & & & & & \\
                    \multicolumn{6}{l}{Chain CRIU} \\
                    target & prot & opt & source & dest & \\
                    ACCEPT & all & -- & .../0 & .../0 & mark 0xc1 \\
                    DROP & all & -- & .../0 & .../0 & \\
                \end{tabular}
                }
                \caption{Output of running \texttt{iptables -t filter -L -n}.\label{table:filter}}.
            \end{table}
        \end{column}
    \end{columns}
\end{frame}

\begin{frame}
    \frametitle{Evaluation}
    \framesubtitle{Perceived Throughput by the Client End}

    \vspace{10pt}

    \textbf{\textcolor{blue}{We drop packets, but what's the impact on performance?}}
    \begin{itemize}
        \item \texttt{iPerf3} client in one VM and \texttt{iPerf3} server in another VM (same LAN).
        \item \texttt{t=0} start client, \texttt{t=10} dump the server, \texttt{t=12} restore the server.
        \item One experiment server as standalone process, another one in a container.
    \end{itemize}

    \vspace{-5pt}

    \begin{figure}
        \centering
        \includegraphics[width=.75\textwidth]{./images/tcp_established_downtime_microbenchmark.pdf}
        \caption{Throughput from the client as a function of time.\label{fig:evaluation-downtime}}
    \end{figure}
    
\end{frame}

\begin{frame}
    \frametitle{Evaluation}
    \framesubtitle{Perceived Throughput by the Client End}

    \vspace{10pt}

    \textbf{\textcolor{blue}{How reactive is to sudden changes?}}
    \begin{itemize}
        \item \texttt{t=0} start client, dump server at \texttt{t=6,10,14} restore immediately.
        \item Where does the difference in downtime comes from?
        \item We don't know! Hidden caching, TCP's congestion and hold back times, \texttt{iPerf3}'s reactivity, ...
    \end{itemize}

    \vspace{-5pt}

    \begin{figure}
        \centering
        \includegraphics[width=.75\textwidth]{./images/tcp_established_resolution_microbenchmark.pdf}
        \caption{Throughput from the client as a function of time.\label{fig:evaluation-downtime}}
    \end{figure}
    
\end{frame}

\begin{frame}
    \frametitle{Problem}
    \framesubtitle{How can we ensure that the same IP can be reused?}

    \begin{columns}
        \begin{column}{.45\textwidth}
            \textbf{\textcolor{blue}{Restore the same IP in a different environment:}}
            \begin{itemize}
                \item Migration within the same machine:
                \begin{itemize}
                    \item Use namespaces and virtual Ethernet pairs (Figure \ref{fig:architecture}).
                    \item Supported by runC containers (use-case particularly suitable).
                    \item Demonstrated in intermediate results presentation.
                \end{itemize}
                \vspace{5pt}
                \item Migration to a remote machine:
                \begin{itemize}
                    \item Two VMs and a third one listening.
                    \item Manually compress and migrate image files.
                \end{itemize}
            \end{itemize}
        \end{column}
        \begin{column}{.55\textwidth}
            \vspace{-20pt}
            \begin{figure}[t!]
                \includegraphics[width=.85\textwidth]{./images/architecture.png}
                \caption{Two different setups for IP reuse.\label{fig:architecture}}
            \end{figure}
        \end{column}
    \end{columns}
    
\end{frame}

\begin{frame}
    \frametitle{Future Work \& Questions}

    \textbf{\textcolor{blue}{Work to-do:}}
    \begin{itemize}
        \item Remote migration with containers still not working.
        \item Once it does, have a similar evaluation comparing CRIU and runC.
    \end{itemize}

    \vspace{15pt}

    \begin{center}
        Thank you for your attention,\\[5pt]

        \Large
        \textbf{\textcolor{blue}{Observations, Doubts \& Suggestions Welcome}}\\[25pt]

        \normalsize
        Follow the development:\\ \url{https://github.com/live-containers/live-migration}\\[15pt]
        \url{carlos.segarra@estudiant.upc.edu}
    \end{center}
\end{frame}

\end{document}
